\begin{frame}
	\frametitle{Proof of Work}
	\framesubtitle{metodo Hashcash}
	
	\begin{itemize}
	  \item \textit{facile} verificare che il messaggio è soluzione di problema \textit{difficile}
	  \item {\color{blue}brute force} unica tecnica risolutiva 
	  \item Problema: dati $h:\mathbb{Z}\rightarrow\mathbb{Z}_n,\,m,\,z \leq n$, trovare nonce $\orange{x}:$
	  		%$$d=h(m|x) < T_z = \sum_{i=1}^{n-z}{2^i}$$ 
	  		$$\orange{d}=\mathrm{hash}^{\mathrm{PoW}}(\orange{m}|\orange{x}) < \orange{T_z} = \orange{2^{n-z+1}}$$
	   		\textit{i.e.} \textbf{digest ha $\orange{z}$ zeri non significativi}
	   		(parametro {\color{blue}target}) $$\Pr[d<T_z|Z=z]=\frac{1}{2^z} \Rightarrow O(\orange{2^z})$$ %SAY dimensione media esponenziale della ricerca
	  \item \fbox{problema risolto $\Leftrightarrow$ blocco $\mathfrak{B}$ risolto $\Leftrightarrow \{\mathfrak{T}\}_\mathfrak{B}$ convalidate}
	  \item Bitcoin: $\;\mathrm{hash}^{\mathrm{PoW}}(\orange{\mathfrak{B}})
	  					=SHA_{256}(SHA_{256}(\;[\orange{\mathrm{Header}}]_{\orange{\mathfrak{B}}}\;))$
	  \end{itemize}
\end{frame}
% ----------------------------------------------------------------------------------------------------------------------------
\begin{frame}
	\frametitle{Proof of Work}
	\framesubtitle{exempli gratia: $z=15$}	
	 
	$$ \mathrm{hash}(\mathrm{``hello\,world''}|\orange{001})=              9002381300129484192947128  $$
	$$ \vdots \;\;\;\;\;\;\;\;\;\;\;\;\;\;\;\;\;\;\;\;\;\;\;\;\;\;\;\;\;\; \vdots$$
	$$ \mathrm{hash}(\mathrm{``hello\,world''}|\orange{034})={\color{red}0000}834716283947104512438 $$
	$$ \vdots \;\;\;\;\;\;\;\;\;\;\;\;\;\;\;\;\;\;\;\;\;\;\;\;\;\;\;\;\;\; \vdots$$
	$$ \mathrm{hash}(\mathrm{``hello\,world''}|\orange{415})={\color{green}00000000000000000000}83201 $$
	\newline
	{\color{blue}n.b.} \textit{gambler's fallacy}
	$$\forall \orange{t_1},\,\orange{t_2} \;\;\; \Pr(Z=\orange{z},\,T=\orange{t_1})=\Pr(Z=\orange{z},\,T=\orange{t_2})$$

\end{frame}
% ----------------------------------------------------------------------------------------------------------------------------
\begin{frame}
	\frametitle{Proof of Work}
	\framesubtitle{adattamento target}
	
	target $z_i$ ricalcolato ogni 2016 blocchi risolti $\sim$ 2 settimane
	\begin{enumerate}
		\item $\Delta_i^t \leftarrow t_{i} - t_{i-1}$
		\item $\Delta_i^t \leftarrow \mathrm{clip}(\Delta_i,\,0.5,\,8)$
		\item $z_{i+1} \leftarrow z_i\;\frac{\Delta_i^t}{2}$
	\end{enumerate}
	\begin{itemize}
		\item $\orange{z} \propto \orange{\Delta^t} \Rightarrow$ {\color{blue}soluzioni veloci} abbassano target 
			\newline $\;\;\;\;\text{\textit{i.e.}}$ generazione {\color{blue}problemi più difficili}, vv.
		\item blocco risolto mediamente ogni 10 minuti
	\end{itemize}
\end{frame}
% ----------------------------------------------------------------------------------------------------------------------------
\begin{frame}
	\frametitle{Proof of Work}
	\framesubtitle{sicurezza di SHA256}
	
	in teoria\ldots
	\begin{itemize}
		\item {\color{blue}preimage} attack: $O(2^{256})$
		\item {\color{blue}birthday} attack: $O(2^{\sfrac{256}{2}})$
	\end{itemize}
	
	\ldots in pratica
	
	\begin{table}
	    \begin{tabular}{l|l|l|l}
		    \textit{Metodo}             & \textit{Attacco}           & \textit{Iterazioni} & \textit{Complessità} \\ \hline
		    deterministico     & collisione        & 24         & $2^{\red{28.5}}$ \\
		    meet in the middle & preimmagine       & 42         & $2^{248.4}$ \\
		    differenziale      & pseudo collisione & 46         & $2^{178}$ \\
		    biclique           & preimmagine       & 45         & $2^{255.5}$ \\
	    \end{tabular}
	\end{table}
\end{frame}
% ----------------------------------------------------------------------------------------------------------------------------
\begin{frame}
	\frametitle{Proof of Work}
	\framesubtitle{prevenzione double spending}
	
	{\color{blue}spendere due volte la stessa $\mathfrak{T}$}
	\begin{enumerate}
		\item modifica $\orange{\mathfrak{I}^{out}}\;$ $\Rightarrow
		\orange{\mathfrak{T}}$ stessa $\Rightarrow \orange{\mathfrak{B}}$ di appartenenza
		\newline\;\;\; ricalcolo nonce $\orange{x}$
		\item modifica $\mathfrak{B} \Rightarrow $ modifica $N$ blocchi successivi
		\newline\;\;\; ricalcolo $N$ nonces $\Rightarrow$ {\color{blue}risolvere $N$ problemi esponenziali}
	\end{enumerate}
	
	\vspace{1pt}
	
	\begin{figure}[H]
	 	\begin{center}
			 \begin{tabular}{c @{\hspace{1em}} c}
				 \includegraphics[width= 11cm]{images/dspending_ppt.png}
			 \end{tabular}
		 \end{center}
 	\end{figure}
 	{\color{blue}forking}
 	\begin{itemize}
 	  \item \textit{policy}: aggiungere sempre a ramo più lungo
 	  \item in una \textit{web of trust} sopravviverà il ramo corretto
 	\end{itemize}	
\end{frame}