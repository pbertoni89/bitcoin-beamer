\documentclass{beamer}
\setbeamertemplate{navigation symbols}{}

\mode<presentation>
{
\usetheme{Warsaw}
\usecolortheme{crane}
\setbeamercovered{transparent}
}

\usepackage[italian]{babel}
\usepackage[utf8]{inputenc}	%[latin1]
\usepackage{patlib}

% font definitions, try \usepackage{ae} instead of the following
% three lines if you don't like this look 
\usepackage{mathptmx}
\usepackage[scaled=.90]{helvet}
\usepackage{courier}
\usepackage[T1]{fontenc}
\usepackage{hyperref}
\usepackage{xfrac}
\usepackage{booktabs}% http://ctan.org/pkg/booktabs
\usepackage{ulem} %for strikeout
\usepackage{xcolor}
%\usepackage[shortlabels]{enumitem} % for letters in enumerate
%\renewcommand{\theenumi}{\Alph{enumi}}

% ----GENERALE-------------------------------------------------------------------

\title{introduzione a Bitcoin}
\subtitle{e alle criptovalute digitali}
\author{P.Bertoni}

% If you have a file called "university-logo-filename.xxx", where xxx
% is a graphic format that can be processed by latex or pdflatex,
% resp., then you can add a logo as follows: 

\pgfdeclareimage[height=0.5cm]{btclogo}{images/bitcoin.png}
%\logo{\pgfuseimage{btclogo}}

% Delete this, if you do not want the table of contents to pop up at
% the beginning of each subsection:
\AtBeginSubsection[]
{
\begin{frame}<beamer>
\frametitle{contenuto}
\tableofcontents[currentsection,currentsubsection]
\end{frame}
}

% If you wish to uncover everything in a step-wise fashion, uncomment
% the following command:
%\beamerdefaultoverlayspecification{<+->}

\begin{document}
% ----TITOLO---------------------------------------------------------------------
\begin{frame}
	\titlepage
\end{frame}
% ----SOMMARIO-------------------------------------------------------------------
\begin{frame}
	\frametitle{Tabella Contenuti}
	\tableofcontents
\end{frame}
% ----SEZIONI--------------------------------------------------------------------
\section{Introduzione}
\subsection{Problema} % %\subsection[Short First Subsection Name]{Problema}
	\begin{frame}

	\frametitle{Criptovaluta}
	\framesubtitle{specifiche sul protocollo}
	
	\textit{Problema}
	\newline implementare una {\color{blue}valuta} economica
	\begin{itemize}
	  \item sicura
	  \item basata su informatica
	  \item decentralizzata % implica distribuita; in senso lato evidenzia l'assenza di nodi master -> SAY da dove viene la fiducia?
	  \item distribuita % l'informazione (o il computo, che è più questo caso) è appunto distribuita su può nodi
	\end{itemize}

\end{frame}

	\begin{frame}
	\frametitle{Criptovaluta}
	\framesubtitle{specifiche sulle transazioni}
	
	\textit{Problema}
	\newline trasmettere {\color{blue}transazioni} di valuta tra enti
	\begin{itemize}
	  \item pubbliche
	  \item anonime $\Rightarrow$ tra \textbf{indirizzi}, non tra utenti
	  \item autenticate
	  \item non ripudiabili
	  \item irreversibili
	\end{itemize}
	registrate in una sorta di {\color{blue} storico} globale
	
\end{frame}
	
\begin{frame}
	\frametitle{Transazione $\mathfrak{T}$}
	\framesubtitle{idea astratta}
	
	\begin{itemize}
	  \item unica modalità di circolazione della valuta
	  \item atto tra $N$ mittenti e $M$ destinatari
	  \item gli enti sono \textit{indirizzi}, \textbf{non} persone!
	  \item utenti incoraggiati a usare un indirizzo unico $\forall$ loro $\mathfrak{T}$
	  \item $\mathfrak{T}$ paragonabile a un {\color{blue}assegno}
	  \newline \textit{``si certifica che, in data $t$,}
	  \newline \textit{$\{x\}$ ha versato} tot \bitcoinA\,\textit{a $\{y\}$, che ora ne è proprietario.''}
	\end{itemize}
	gestione del {\color{blue}resto}
	\begin{itemize}
		\item $y'\in\{y\}$ destinatari, indirizzo controllato da chi emette $\mathfrak{T}$
		\item diversi resti frammentati sono riuniti come molteplici mittenti 
	\end{itemize}
\end{frame}
	\begin{frame}

	\frametitle{Criptovaluta}
	\framesubtitle{specifiche sull'affidabilità}
	
	\textit{Problema}
	\newline progettare un algoritmo di {\color{blue}mining} per convalidare transazioni
	\begin{itemize}
	  \item \textit{trattabile} da decidere
	  \item \textit{intrattabile} da risolvere
	  \item dipendente da lista transazioni in attesa
	\end{itemize}
	motivazione a {\color{blue}partecipare} al mining
	\begin{itemize}
		\item ricompense
		\begin{itemize}
		  \item immediate
		  \item durevoli
		\end{itemize}
	  \item complessità lavoro onesto $\equiv$ complessità disonesto 
	  \item ma $\Pr[\text{successo lavoro disonesto}]\rightarrow 0$
	\end{itemize}
	
\end{frame}

	
\subsection{Caratteristiche di una criptovaluta}
	\begin{frame}
	\frametitle{Differenze con valute tradizionali}
	
\begin{table}[h]
\begin{tabular}{ll}
	%\hline
	%\multicolumn{1}{c}{} &
	%\multicolumn{1}{c}{\justfig{euro.png}{0.1}} \\
	\multicolumn{1}{c}{\includegraphics[width = 1 cm]{images/euro.png}} &
	\multicolumn{1}{c}{\includegraphics[width = 1 cm]{images/bitcoin.png}} 											  \\
	
	\multicolumn{2}{l}{{\color{blue}ontologia}}                                                               \\
	\multicolumn{1}{c}{esplicita: come unità fisiche} &
	\multicolumn{1}{c}{implicita: in funzione di transazione}															  \\ \hline \\
	
	\multicolumn{2}{l}{{\color{blue}fiducia nell'accettazione di moneta}}                                             \\
	\multicolumn{2}{c}{difficoltà di contraffazione}                                                                  \\ \hline \\
	
	\multicolumn{2}{l}{{\color{blue}possibilità di furto o smarrimento}}			\\
	\multicolumn{1}{c}{gettone fisico} & \multicolumn{1}{c}{chiave privata di firma digitale}                         \\ \hline \\
	\multicolumn{2}{l}{{\color{blue}fiducia nel protocollo di supporto}}                                              \\
	\multicolumn{1}{c}{ente nazionale o sovranazionale} & \multicolumn{1}{c}{modello formale di sicurezza}            \\
	\end{tabular}
\end{table}

\end{frame}
	\begin{frame}
	\frametitle{i primordi: eCash [Chaum]}
	\framesubtitle{sistema di firma digitale a \textit{conoscenza zero}}
	
	Scenario (\textit{e.g.} voto digitale)
	\begin{itemize}
	  \item sia $\orange{m}$ plaintext, $A$ autore e $F$ firmatario,\;\;$A\neq F$
	  \item firma \textit{classica}: $S(\orange{m},\,\orange{K_{PR}^A})$
	  \item firma \textit{{\color{blue}cieca}}: $S^\red{B}(\red{f}(\orange{m}),\,\orange{K}_{\orange{PR}}^\red{F})$
	  \item $F$ non può calcolare $\orange{m}$ data $f(\orange{m})$  
	\end{itemize}
	
	Algoritmo
	\begin{enumerate}
	  \item $A$ estrae \textit{nonce} $\orange{x}$
	  \item $A$ invia $\orange{\overline{m}}=f(\orange{m},\,\orange{x})$ messaggio \textit{cieco} a $F$
	  \item $F$ restituisce $\orange{s^B}=\orange{\overline{s}}=E(\orange{\overline{m}},\,\orange{K_{PR}^F})$ 
	  \item chiunque conosca $\orange{K_{PB}^F}$ calcola $\orange{\overline{m}^S}=E(\orange{\overline{s}},\,\orange{K_{PB}^F})$
	  \item \hspace{1pt}\;\;\;\;\;\;``\;\;\;\;\;\;\;\;\;\;\;\;\; ``\;\;\;\;\;\; pure $\orange{x}$ \;\;\;\;''\;\; $\orange{m^S}=f^{-1}(\orange{\overline{m}^S},\,\orange{x})$
	\end{enumerate}

\end{frame}
\subsection{Elementi teorici}
	\begin{frame}
	\frametitle{Curve Ellittiche}

\begin{columns}
	\begin{column}{.65\textwidth}

		\begin{itemize}
			\item curve definite su un certo $\mathbb{F}_\orange{q}$ da 
				$$y^2=x^3+\orange{a}x+\orange{b}$$
			\item non singolari, \textit{i.e.} $4a^3+27b^2\neq 0$
		\end{itemize}

		\begin{theorem}[Hasse]
			 sia $\mathbb{F}_q$ il campo di Galois di ordine $q$ 
			 \newline sia $\mathcal{E}_q=\mathcal{E}_{(a,b)}(\mathbb{F}_q)$ una sua curva ellittica 
			\vspace*{4pt}
			\begin{enumerate}
				\item $|o(\mathcal{E}_q)-(q+1)|\leq2\sqrt q$
			\end{enumerate}	
		\end{theorem} 
		$\;\;\;\Rightarrow\;\;${\color{blue}ordine} \textit{GF} governa \textit{difficoltà}

	\end{column}

	\begin{column}{.4\textwidth}
		\includegraphics[height = 5 cm]{images/eca.png}
	\end{column}
\end{columns}

\end{frame}
%-------------------------------------------------------------------------
\begin{frame}
	\frametitle{Curve Ellittiche}
	\framesubtitle{legge di gruppo: definizione}
	
	$(\mathcal{E}_{(a,b)}(\mathbb{F}_q),\; \red{+})$ definisce un {\color{blue}gruppo abeliano}
	$$ \orange{R} = \orange{P}+\orange{Q} \triangleq(\orange{x_R},\,{\color{red}-}\orange{y_R}) $$
	$$x_P \neq x_Q \;: \;\;
  			\left \{ \begin{array}{lr}
	  			y_R \triangleq y_P+s(x_R-y_P) \\
				x_R \triangleq s^2-x_P-x_Q & s=\frac{y_P-y_Q}{x_P-x_Q}
			\end{array} \right. $$
			
	$$x_P = x_Q \;: \;\;
  			\left \{ \begin{array}{lcr}
 
  			y_P = -y_Q\;: & R = O \\
			 y_P = y_Q \neq 0\;: & \left\{
				  \begin{array}{lcr}
				  	y_R \triangleq y_P+s(x_R-y_P) \\
				    x_R \triangleq s^2-2x_P & s=\frac{3x_P^2+a}{2y_P}
				  \end{array}
				\right. 
			\end{array} \right. $$
 	\vspace{1pt}				
 	$$ \orange{R} = \orange{P}{\color{red}\times}\orange{n} 
 		\triangleq \orange{P}+\orange{P}+\ldots+\orange{P}\;\;\;\;\;\;\;\;\; n\in \mathbb{Z}\;\;\mathrm{volte}$$
	
\end{frame}
%-------------------------------------------------------------------------
\begin{frame}
\frametitle{Curve Ellittiche}
	\framesubtitle{legge di gruppo: casistica}
	
	\begin{figure}[H]
	 	\begin{center}
			 \begin{tabular}{c @{\hspace{1em}} c}
				 \includegraphics[width = 11 cm]{images/ecalgebrarow.png}
			 \end{tabular}
		 \end{center}
 	\end{figure}
	
\end{frame}	
%-------------------------------------------------------------------------
\begin{frame}
	\frametitle{Curve Ellittiche}
	\framesubtitle{problema matematico}
	
	%$(\mathcal{E}_{(a,b)}(\mathbb{F}_q),\,+)$ è un gruppo abeliano
% 	\begin{itemize}
% 		\item chiusura
% 		\item associatività
% 		\item identità %SAY come visto prima
% 		\item invertibilità %SAY come visto prima
% 		\item commutatività
% 	\end{itemize}  
	
	trovare un segreto ${\color{red}d}\in[1,\,n-1]$, dati
	\begin{itemize}
	  \item $\orange{\mathcal{E}}=\mathcal{E}_{(a,b)}(\mathbb{F}_q)$
	  \item $\orange{G}\in \mathcal{E}:\;\;\;\;\;\;\tiny{<}\,G\,\tiny{>}=\mathcal{E}$
	  \item $\orange{n}=o(G):\;G\times n= O=P_\infty$, $\;\;	\;n\;$ primo
	  \item $\orange{P}\in \mathcal{E}$
	  \item $\orange{Q}=P\times d$
	\end{itemize}
\end{frame}
	\begin{frame}
	\frametitle{Funzioni di Hash}

	\begin{itemize}
		\item $\orange{h}:\orange{\mathbb{Z}}\rightarrow \orange{\mathbb{Z}_n}$ non iniettiva, \textit{i.e.} one-way

		\item resistenza a
		\begin{itemize}
			\item preimmagine $\rightarrow$ ricerca \textit{bruta} è $O(\orange{2^n})$
			\item collisioni deboli $\rightarrow$ \;\;\;\;\; ''\;\;\;\;\; ''\;\;\;\;\; ''
			\item collisioni forti $\rightarrow$ \textit{birthday}: ricerca \textit{bruta} è $O(\orange{2^{\sfrac{n}{2}}}) \ll O(2^n)$
		\end{itemize}
		
		\item usate soprattutto per
		\begin{itemize}
		  \item autenticazione
		  \item integrità
		\end{itemize}
		\item spesso viene firmato il {\color{blue}digest} $d=\orange{h}(\orange{m})$ anzichè $\orange{m}$
		\newline nei sistemi di autenticazione e non ripudio
		\item insieme ai \textit{salt} prevengono attacchi \textit{dizionario}
	\end{itemize}

\end{frame}
	\begin{frame}
	\frametitle{Alberi di Merkle}
	
\begin{columns}
 \begin{column}{.72\textwidth}

		\begin{itemize}
			\item foglie $\leftrightarrow$ transazioni
			\item altri nodi $\leftrightarrow$ hash dei loro figli
			\item dimostrare che una foglia $\in$ albero: $O(\orange{\log_2 N})$
			\item usando una lista: $O(N) \gg O(\log_2 N)$
			\item protegge integrità transazioni
			\item nati per firme \textit{one time} di \textit{Lamport}
			\item Bitcoin: $\;hash_n=SHA_{256}(SHA_{256}(\mathcal{L}_n|\mathcal{R}_n))$
		\end{itemize}

	\end{column}

	\begin{column}{.4\textwidth}
		\includegraphics[height = 4 cm]{images/merkle.png}
	\end{column}
\end{columns}

\end{frame}
% -------------------------------------------------------------------------
\section{Protocollo}
\subsection{Strutture dati}
	\begin{frame}
	\frametitle{Transazione $\mathfrak{T}$}
	\framesubtitle{struttura dati}
	
	\begin{itemize}
		\item ogni mittente appone la propria firma 
		\item $\sum_i^N \orange{\mathbb{B}_i^{in}} \geq \sum_i^M \orange{\mathbb{B}_i^{out}}$
	\end{itemize}
	
	\begin{table}
	  \centering
	  \begin{tabular}{l}
	    \midrule \midrule
	    $\bullet\;\orange{\mathfrak{T}^{ID}}=h(\mathfrak{T})$\\
	    $\bullet\;$ timestamp $\orange{t}$ \\
	    \begin{tabular}{l|l}
			$\bullet\;\forall i$ indirizzo di {\color{blue}input}  $\orange{\mathfrak{I}_i^{in}}$ & 
			$\bullet\;\forall j$ indirizzo di {\color{blue}output} $\orange{\mathfrak{I}_j^{out}}$ \\
			\tabitem \sout{somma trasferita $\mathbb{B}_i^{in}$} & \tabitem somma ricevuta $\orange{\mathbb{B}_j^{out}}$ \\
			\tabitem chiave $\orange{K^{PB}_{i,\,in}}$ & \tabitem chiave $\orange{K^{PB}_{j,\,out}}$ \\
			\tabitem indice $\orange{p}: \mathfrak{I}_i^{in}=\mathfrak{I}_p^{\prime out}$ \\
			\tabitem $\orange{\mathfrak{T}^{ID}_{-1}}=h(\mathfrak{T}_{-1})$ precedente \\
			\tabitem firma $\orange{S_{i}}(h(\orange{\mathfrak{T}_{-1}}, \orange{K^{PB}_{j,\,out}})$	
		\end{tabular}		
	  \end{tabular}
	\end{table}

\end{frame}
	\begin{frame}
	\frametitle{Blocco $\mathfrak{B}$}
	\framesubtitle{struttura dati} 
	
	\begin{table}
	  \centering
	  \begin{tabular}{ll}
	   	{\color{blue}Header} & {\color{blue}Payload} \\
			\tabitem hash di $\orange{\mathfrak{B}_{i-1}^{\mathfrak{H}}}$ 
				& \tabitem lista transazioni $\{\orange{\mathfrak{T}}\}_{\orange{\mathfrak{B}}}$ \\
			\tabitem MerkleTree di $\{\orange{\mathfrak{T}}\}_\orange{\mathfrak{B}}$ \\
			\tabitem timestamp $\orange{t}$ \\
			\tabitem target $\orange{z}$ \\
			\tabitem nonce $\orange{x}$ \\
			\tabitem titolare della \textit{coinbase} $\orange{\mathfrak{C}}$
	  \end{tabular}
	\end{table}
	
	\begin{itemize}
		\item durante \textit{mining} di $\mathfrak{B}$, campi continuamente modificati
		\item $\mathfrak{B}$ descrive la propria \textit{proof of work}
	\end{itemize}

\end{frame}
	\begin{frame}
	\frametitle{Transazioni $\rightarrow$ Blocchi} 
	\framesubtitle{generazione nuova transazione}
	
	\begin{enumerate}
		\item broadcastata tramite protocollo \textit{flooding}
		\item ogni miner \textbf{può} includerla nel suo \textit{pool}
		\item inizialmente inserita in un pool come \textit{invalida}
		\item dopo risoluzione del $\mathfrak{B}$ corrente è rimossa da ogni pool
	\end{enumerate}

	\begin{figure}[H]
		\begin{center}
			\includegraphics[height = 4.5 cm]{images/chain_transactions.png}	
		\end{center}
	\end{figure}
\end{frame}
	\begin{frame}
	\frametitle{Blocchi $\rightarrow$ Blockchain} 
	\framesubtitle{generazione catena di blocchi}
	
		\begin{itemize}
			\item hash dei blocchi precedenti \vspace{1pt} $\sim$ \vspace{1pt} puntatori di una lista
			\item a ritroso si giunge al $\mathfrak{B}$ di \textit{genesi}
		\end{itemize}
		
	 	\begin{figure}[H]
			\begin{center}
				\includegraphics[height = 4.5 cm]{images/blockchain.png}	
			\end{center}
		\end{figure}

\end{frame}
	%\input{frames/motivazione_mining} %deve essere invitante lavorare per la rete, non contro essa
	\begin{frame}
	\frametitle{Motivazione all'utilizzo}
		\framesubtitle{transazioni Coinbase $\mathfrak{C}$}
		
		\begin{itemize}
			\item $\forall \orange{\mathfrak{B}},\; \exists!\:\orange{\mathfrak{C}}$
			\item inputs: $\orange{\emptyset}$
			\item outputs: ricompensa a {\color{blue}miners risolutori} di $\mathfrak{B}$ 
			\begin{itemize}
				\item \textit{coinbase}
				\begin{itemize}
				  	\item \orange{50} \bitcoinA\; iniziali
					\item dimezzata ogni 210K blocchi risolti
					\item nulla dopo 6.93M blocchi
				\end{itemize}
				\item $\{\orange{\mathfrak{F}}\} \in \orange{\mathfrak{B}}$
			\end{itemize}
			\item inflazione 
			\begin{itemize}
				\item dettata solo da mining
				\item limitata
					$$\sum_{i=0}^{6.93M-1}{\frac{50}{2^{\lfloor\sfrac{i}{210K}\rfloor}}}=\orange{21M}\;\bitcoinA $$ 	
			\end{itemize}
		\end{itemize}
		
\end{frame}
 
% In addition to the newly created Bitcoins, the coinbase transaction is also used for assigning the recipient of any transaction fees 
% that were paid within the other transactions being included in the same block. 
% 
% The coinbase transaction can assign the entire reward to a single Bitcoin address, 
% or split it in portions among multiple addresses, just like any other transaction. 
% 
% Coinbase transactions always contain outputs totaling 
% the sum of the block reward plus all transaction fees collected from the other transactions in the same block.
% The coinbase transaction in block zero cannot be spent. 
% This is due to a quirk of the reference client implementation that would open the potential for a block chain fork 
% if some nodes accepted the spend and others did not[1].
% capito: significa che sicuramente nel primo blocco han circolato, ma dal secondo è come se si fossero congelati e divenuti inutilizzabili

	\begin{frame}
	\frametitle{Motivazione all'utilizzo}
		\framesubtitle{fees $\mathfrak{F}$ sulle transazioni}
		
		$\forall$ transazione $\orange{\mathfrak{T}}$
		\begin{itemize}
			\item $\orange{\mathfrak{F}} = \sum_i^N \orange{\mathbb{B}_i^{in}} - \sum_i^M \orange{\mathbb{B}_i^{out}} \geq 0$
			\item spetta a miners che risolvono $\mathfrak{B}\ni\mathfrak{T}$
			\item $ \left \{
					  \begin{array}{lcr}
					    \text{\textbf{mai} obbligatoria, ma\ldots}\\ %SAY d'altronde
					    \text{miners \textbf{mai} obbligati ad aggiungere}\:\mathfrak{T}\:\text{a proprio pool di lavoro}  %SAY l'importante è risolvere il blocco, per loro un problema vale l'altro
					  \end{array}
					\right. $  
		\end{itemize}
		\vspace{5pt}
		$  \;\;\; \Rightarrow$ fees {\color{blue}incentivi} per
		\begin{itemize}
			\item velocizzare validazione di $\mathfrak{T}$
			\item \textit{mining} costante nonostante decrescita \textit{coinbase rewards}
		\end{itemize}

\end{frame}
\subsection{Primitive criptografiche} %forse, ma non credo, vanno in testa alla sezione
	% DOMANDE TEORICHE:
% 1 - DIMOSTRARE CORRETTEZZA FIRMA
% 2 - DIMOSTRARE ATTACCO CON STESSO K

\begin{frame}
	\frametitle{ECDSA}
	\framesubtitle{inizializzazione}

{\color{blue}Alice}: scelta dei parametri \textbf{pubblici}
	\begin{enumerate}
	  \item $\orange{q}=2^\orange{m}$
	  \item $(\orange{a},\,\orange{b}):\;4a^3+27b^2\neq 0$
	  \item ${\color{orange}G}=(x_G,\,y_G) \in \mathcal{E}_{(a,b)}(\mathbb{F}_q)$ 
	  \item ${\color{orange}n}=o(G)$
	\end{enumerate}
{\color{blue}Alice}: generazione coppia chiavi
	\begin{enumerate}
	  \item $K^{PR}\triangleq {\color{orange}d_A}\leftarrow \mathrm{rand} \in[1,\,n-1]$
	  \item $K^{PB}\triangleq {\color{orange}Q_A} \leftarrow {\color{orange}n}\times {\color{orange}d_A}$
	\end{enumerate}
{\color{blue}Bob}: verifica validità di $Q_A$ ricevuta
	\begin{enumerate}
	  \item $Q_A\neq O$
	  \item $Q_A\in \mathcal{E}$
	  \item $Q_A\times n=O$
	\end{enumerate}
\end{frame}
% -----------------------------------------------------------------------------
\begin{frame}
\frametitle{ECDSA}
\framesubtitle{firma digitale}

	{\color{blue}Alice}: firma del messaggio ${\color{orange}m}$ 
	\begin{enumerate}
	  \item $e \leftarrow \mathrm{hash}(m)$  %se scommento sotto, diventa e
	  % SOLO SU WIKI LETTA \item $z \leftarrow \lceil\log_2 n\rceil$ bit più a sinistra di $e$
	  \item $k \leftarrow \mathrm{rand} \in [1,n] \subset \mathbb{N}$
	  \item $(x_1,y_1) \leftarrow k \times G$
	  \item $r \leftarrow x_1\mod n$ %\\ $\;\;\;\;$ 
	  \item \textbf{if} $r=0$ \textbf{goto} 2
	  \item $s \leftarrow k^{-1}(e+rd) \mod n$ %\\ $\;\;\;\;$ 
	  \item \textbf{if} $s=0$ \textbf{goto} 2
	  \item \textbf{return} $({\color{orange}r},\,{\color{orange}s})$
	\end{enumerate}
\end{frame}
% -----------------------------------------------------------------------------
\begin{frame}
\frametitle{ECDSA}
\framesubtitle{verifica firma}	

	{\color{blue}Bob}: verifica firma $(r,s)$ di $m$
	\begin{enumerate}
	  \item $(r,\,s)\in [1,\,n-1]\times[1,\,n-1]$
	  \item $e \leftarrow \mathrm{hash}(m)$  %se scommento sotto, diventa e
	 	%SOLO SU WIKI LETTA \item $z \leftarrow \lceil\log_2 n\rceil$ bit più a sinistra di $e$
	  \item $w=s^{-1} \mod n$ 
	  \item $(u_1,\,u_2) \leftarrow (ew\mod n,\;rw \mod n)$
	  \item $(x_1,\,y_1) \leftarrow u_1 \times G + u_2 \times Q$
	  \item \textbf{ok} $\Leftrightarrow \orange{r} \equiv \orange{x_1} \mod \orange{n}$
	\end{enumerate}
\end{frame}
	\begin{frame}
	\frametitle{SHA-256}
	\framesubtitle{Secure Hash Algoritm 256 bit}
	
	\begin{itemize}	 
			\item $\sfrac{256}{2}=128$ bit di sicurezza $\Rightarrow$ collisioni non ancora trovate 
			\item $\mathrm{len}(m)\leq 2^{64}-1$
	\end{itemize}

	\begin{columns}
	 \begin{column}{.65\textwidth}
		\begin{enumerate}	 
		  	\item padding: $\mathrm{len}(m|0\dots)\equiv448\mod512$
			\item padding con 64 bit di $\mathrm{len}(m)$  
			\item $N$ blocchi da 512 bit: $\; B^{(1)},\,\dots B^{(N)} $
			\item $$ H^{(i)} \triangleq H^{(i-1)}+C_{B^{(i)}}(H^{(i-1)})$$
			\item ritorna $d=H^{(N)}$
		\end{enumerate}
	 \end{column}
	
	 \begin{column}{.55\textwidth}
	 	\begin{figure}
		 	\includegraphics[height = 4 cm]{images/sha2.png}
		 	\caption{funzione di compressione $C$}
	 	\end{figure}
	 \end{column}
	\end{columns}

\end{frame}
\subsection{Modello formale di sicurezza}
	\begin{frame}
	\frametitle{Firma digitale}
	
	\begin{itemize}
		\item $|K_{ECC}|=256$ bit $\simeq |K_{RSA}|=3072 $ bit
		\item $\orange{q}\simeq 10^{\orange{77}},\;\;\orange{n}\simeq 10^{\orange{69}},
				\;\;\mathcal{E}:y^2=x^3+\orange{0}x+\orange{7}$
	\end{itemize}
	
	\begin{itemize}
	  	\item {\color{blue} meet in the middle} [Shank]: $\;\;\;\;\Omega(\sqrt{\orange{q}})$
		\item nonce $\orange{k}$ è confidenziale: $\;\;\;\;d=\red{r}^{-1}(k\red{s}-\red{e})$
		\item {\color{blue} replay attack:} nonce deve tale
		\begin{enumerate}%[label=\Alph*]
		  	\item[a)] $r_1=r_2=r$
			\item[b)] $\red{s_1}\equiv k^{-1}(\red{e_1}+d\red{r})\mod \red{n}$ 
					  $,\;\;\;\; \red{s_2}\equiv k^{-1}(\red{e_2}+d\red{r})\mod \red{n}$
			\item[c)] $k(s_1-s_2)\equiv(e_1-e_2)\mod n$
			\item[d)] $m_1\neq m_2 \Rightarrow (s_1-s_2)\neq 0\Rightarrow
			k\equiv(s_1-s_2)^{-1}(e_1-e_2)\mod n$
		\end{enumerate}
	\end{itemize} 

\end{frame}

	\begin{frame}
	\frametitle{Proof of Work}
	\framesubtitle{metodo Hashcash}
	
	\begin{itemize}
	  \item nato per contrastare spam, DoS: serve un'operazione onerosa
	  \item \textit{facile} verificare che il messaggio è soluzione di problema \textit{difficile}
	  \item {\color{blue}brute force} unica tecnica risolutiva 
	  \item Problema: dati $h:\mathbb{Z}\rightarrow\mathbb{Z}_n,\,m,\,z \leq n$, trovare nonce $x:$
	  		%$$d=h(m|x) < T_z = \sum_{i=1}^{n-z}{2^i}$$ 
	  		$$d=\orange{h}(\orange{m}|\orange{x}) < T_z = \orange{2^{n-z+1}}$$
	   		\textit{i.e.} \textbf{digest ha $z$ zeri non significativi} (parametro {\color{blue}target})
	  		$$\Pr[d<T_z|Z=z]=\frac{1}{2^z} \Rightarrow \orange{O(2^z)}$$ %SAY dimensione media esponenziale della ricerca
	  \item problema risolto $\Leftrightarrow$ blocco $\mathfrak{B}$ risolto $\Leftrightarrow \{\mathfrak{T}\}_\mathfrak{B}$ convalidate  
	  \end{itemize}
\end{frame}
% ----------------------------------------------------------------------------------------------------------------------------
\begin{frame}
	\frametitle{Proof of Work}
	\framesubtitle{exempli gratia: $z=15$}	
	 
	$$ \mathrm{hash}(\mathrm{``hello\,world''}|001)=              9002381300129484192947128  $$
	$$ \vdots \;\;\;\;\;\;\;\;\;\;\;\;\;\;\;\;\;\;\;\;\;\;\;\;\;\;\;\;\;\; \vdots$$
	$$ \mathrm{hash}(\mathrm{``hello\,world''}|034)={\color{red}0000}834716283947104512438 $$
	$$ \vdots \;\;\;\;\;\;\;\;\;\;\;\;\;\;\;\;\;\;\;\;\;\;\;\;\;\;\;\;\;\; \vdots$$
	$$ \mathrm{hash}(\mathrm{``hello\,world''}|415)={\color{green}00000000000000000000}83201 $$
	\newline
	{\color{blue}n.b.} \textit{gambler's fallacy}
	$$\forall t_1,t_2 \;\;\; \Pr(Z=z, T=t_1)=\Pr(Z=z, T=t_2)$$

\end{frame}
% ----------------------------------------------------------------------------------------------------------------------------
\begin{frame}
	\frametitle{Proof of Work}
	\framesubtitle{adattamento target}
	
	\begin{itemize}
		\item target $z_i$ ricalcolato ogni 2016 blocchi risolti $\sim$ 2 settimane
		\item $\Delta_i \leftarrow t_{i} - t_{i-1}$
		\item $\Delta_i \leftarrow \mathrm{clip}(\Delta_i,\,0.5,\,8)$
		\item $z_{i+1} \leftarrow z_i\;\frac{\Delta_i}{2}$
		\item $z \propto \Delta \Rightarrow$ {\color{blue}soluzioni veloci} abbassano target 
			\newline $\;\;\;\;\text{\textit{i.e.}}$ generazione {\color{blue}problemi più difficili}, vv.
		\item blocco risolto mediamente ogni 10 minuti
	\end{itemize}
\end{frame}
% ----------------------------------------------------------------------------------------------------------------------------
\begin{frame}
	\frametitle{Proof of Work}
	\framesubtitle{sicurezza di SHA256}
	
	in teoria\ldots
	\begin{itemize}
		\item {\color{blue}preimage attack} attack: $O(2^{256})$
		\item {\color{blue}birthday attack}: $O(2^{\sfrac{256}{2}})$
	\end{itemize}
	
	\ldots in pratica
	
	\begin{table}
	    \begin{tabular}{l|l|l|l}
		    \textit{Metodo}             & \textit{Attacco}           & \textit{Iterazioni} & \textit{Complessità} \\ \hline
		    deterministico     & collisione        & 24         & $2^{28.5}$ \\  %TODO approfondire
		    meet in the middle & preimmagine       & 42         & $2^{248.4}$ \\ %TODO approfondire
		    differenziale      & pseudo collisione & 46         & $2^{178}$ \\
		    biclique           & preimmagine       & 45         & $2^{255.5}$ \\
	    \end{tabular}
	\end{table}
\end{frame}
% ----------------------------------------------------------------------------------------------------------------------------
\begin{frame}
	\frametitle{Proof of Work}
	\framesubtitle{prevenzione double spending}
	
	per spendere due volte la stessa $\mathfrak{T}$ occorre
	\begin{itemize}
		\item modificarne gli output $\Rightarrow \mathfrak{T}$ stessa $\Rightarrow \mathfrak{B}$ di appartenenza
		\item ricalcolare nonce $x$ per $\mathfrak{B}$ modificato
		\item modifica $\mathfrak{B} \Rightarrow $ modifica di $N$ blocchi successivi
		\item ricalcolare $N$ nonces $\Rightarrow$ risolvere $N$ problemi esponenziali
	\end{itemize}
	
	\vspace{1pt}
	
	\begin{figure}[H]
	 	\begin{center}
			 \begin{tabular}{c @{\hspace{1em}} c}
				 \includegraphics[width= 11cm]{images/dspending_ppt.png}
			 \end{tabular}
		 \end{center}
 	\end{figure}
 	{\color{blue}forking}
 	\begin{itemize}
 	  \item risolto imponendo aggiunta a ramo più lungo
 	  \item sotto ipotesi \textit{web of trust} sopravviverà il ramo corretto
 	\end{itemize}	
\end{frame}
	\begin{frame}
	\frametitle{Web of trust}
	
	dato un pool di miners $\orange{\mathfrak{M}}$ con capacità di calcolo $\orange{\mathcal{C}_\mathfrak{M}} \;\;[\sfrac{\mathrm{GH}}{\mathrm{s}}] $ 
	
	$$\Pr[\mathfrak{M} \mathrm{\;risolva\;blocco}] \propto \frac{\orange{\mathcal{C}_\mathfrak{M}}}{\orange{\mathcal{C}_{\Omega}}} $$
	
	$\Rightarrow$ {\color{blue} condizione necessaria Bitcoin}: \:$\orange{\mathcal{C}_{\mathrm{fair}}} \geq 50\%\;\orange{\mathcal{C}_{\Omega}} $ 
	\newline \newline
	se $\exists\;\orange{\mathfrak{M}_{\mathrm{unfair}}}$ pool disonesto \textit{t.c.} 
		$\orange{\mathcal{C}_{\mathfrak{M}_{\mathrm{unfair}}}} \geq 50\%\;\orange{\mathcal{C}_{\Omega}} $ 
	$$\Rightarrow \text{catena più lunga comanda} \Rightarrow \text{crollo fiducia} \Rightarrow \text{crollo valore}$$
	\begin{itemize}
		\item no motivazione diretta di lucro
		\item ma problema irrisolvibile perchè sistematico
	\end{itemize}
	
\end{frame}

% -------------------------------------------------------------------------
\section{Attacchi tipici}
\subsection{Forgiatura}
	\begin{frame}
	\frametitle{forgiatura}
	
	unica possibilità: rubarli a qualcuno
	\begin{itemize}
	  \item problema di forgiare dal nulla \textbf{non ha senso}
	  \item  conoscere $\orange{K_{PR}}\;\Rightarrow$ rompere ECDSA
	\end{itemize}
	\vspace{7pt}
	se {\color{blue}quantum computers} implementati 
	  \begin{itemize}
	    \item rottura ECDSA \textit{può} diventare \textit{facile}
	    \item collisione $\mathrm{SHA}_{256}$ resta \textit{difficile}
	  	\begin{itemize}
	  		\item indirizzi $\;\orange{\mathfrak{I}} = \mathrm{SHA}_{256}(\mathrm{SHA}_{256}(\orange{K_{PB}}))$
		    \item ottenere $K_{PB}$ dal solo $\mathfrak{I}$ è \textit{difficile}
		    \item ma se è nota sistema rotto
		\end{itemize}
	  \end{itemize}
\end{frame}
	\begin{frame}
\frametitle{caso di studio: Mt.GoX [2014]}
\framesubtitle{prima del crollo}
	
	{\color{blue} exchange}
	\begin{itemize}
	  \item web service di scambio valute \textit{fiat} con \bitcoinA
	  \item 2011: compromissione account interno al perimetro $\Rightarrow$ furto \bitcoinA
	  \item 2013: Blockchain fork in rami con regole diverse 
	  \newline $\;\;\;\;\Rightarrow$ MtGox sospende transazioni
	\end{itemize}
	
	{\color{blue} malleabilità}
	\begin{itemize}
		\item $\mathfrak{T}^{ID}\triangleq h(\mathfrak{T}),\;\;\;S\in\mathfrak{T}$
	 	\item codice \textit{MtGox} non controlla la validità del formato di $S$
	 	\item problema $\in$ implementazione, $\mathbf{\notin}$ protocollo \bitcoinA
	 	\item $\exists$ altri sistemi migliori per proprio storico $\Rightarrow$ immunità
	\end{itemize}

\end{frame}

\begin{frame}
\frametitle{caso di studio: Mt.GoX [2014]}
\framesubtitle{il crollo: attacco delle transazioni \textit{mutanti}}

	{\color{blue} frode} di \textit{Eve} ai danni di \textit{MtGox}
 	\begin{enumerate}
 	  	\item $M$ invia $\orange{\mathfrak{T}}$ a $E$ dopo legittima richiesta prelievo
	 	\item $E$ ritocca $\orange{S}(\orange{\mathfrak{T}})$ prima della sua conferma
	 	\item ora $\exists\,\orange{\tilde{\mathfrak{T}}}\equiv\orange{\mathfrak{T}},\;$ 
	 			ma $\orange{h}(\orange{\tilde{\mathfrak{T}}})\neq \orange{h}(\orange{\mathfrak{T}}) \Rightarrow 
	 			\orange{\tilde{\mathfrak{T}}^{ID}}\neq\orange{\mathfrak{T}^{ID}}$ 
	 	\item $\orange{\tilde{\mathfrak{T}}}$ diffusa da $E$ e confermata prima di $\orange{\mathfrak{T}}$
	 	\item $\orange{\mathfrak{T}}$ non confermata perchè $\orange{\tilde{\mathfrak{T}}}$ invalido
	 	\item $E$ manda un complain a $M$ per $\orange{\mathfrak{T}}$ non ricevuta
	 	\item $M$ controlla il suo storico: $\orange{\mathfrak{T}}$ non è stata accettata
	 	\item $M$ costretto a inviare $\orange{\mathfrak{T}^\prime}$ come rimborso
 	\end{enumerate}

	{\color{blue} DDoS}
	\begin{enumerate}
	  	\item $M$ riceve tante transazioni mutanti
	  	\item grande sovraccarico, persino se c'è resistenza a frode
	  	\item latenza nelle risposte $\Rightarrow$ incertezza $\Rightarrow$ speculazione
	\end{enumerate}

\end{frame}

% questa è L'ALTRA VERSIONE.
% NON tutte le cose utilizzate nell'hash sono anche firmate
% è possibile modificare una T in modo che l'hash ovviamente cambi, 
% ma la firma resti valida!!!!
% non è quindi safe accettare una catena di transazioni NON confermate

%This would have left Mt. Gox's ledgers increasingly out of balance with the public blockchain record.

% More importantly, the integrity of the Bitcoin network itself was never under threat. 
% TXIDs are used only for tracking bitcoin transfers by second-layer software, and their malleability has no impact on the actual transfer of bitcoin. 
% This is why the issue has not been a high priority for repair since it came to developers' attention in 2011.
% "The core software itself, we made zero changes to it, and we plan to make zero changes to it," says Garzik. 
%While the price of bitcoin staggered immediately following Mt. Gox's announcement, it rebounded as the limited nature of the problem became clear, 
%and has remained stable since -- though still at about 50\% of its early-December high.

% MENEZES!!!!!!
% A public-key encryption scheme is said to be non-malleable if given a ciphertext, 
% it is computationally infeasible to generate a different ciphertext such that the respective plaintexts are related in a known manner.
% If a public-key encryption scheme is non-malleable, it is also semantically secure.



\subsection{Anonimità}
	\begin{frame}
	\frametitle{caso di studio: Reid [2011]}
	\framesubtitle{rete transazioni $\mathcal{T}$}
	
	% IDEA: partire da un'informazione pubblica, la blockchain, per associare
	% transazioni agli utenti
	
	% vertice == transazione t
	% arco diretto == flusso di BTC da un output di t a un input di t'
	
	% non è complesso costruire la rete: sono tutti dati pubblici
	% sono omesse transazioni non collegate ad altre transazioni, i.e. coinbase o fees non riscosse
	
	% il grafo è sicuramente un DAG; una transazione non può chiudersi su sè stessa!!!
	\begin{figure}[H]
	 	\begin{center}
			 \begin{tabular}{c @{\hspace{1em}} c}
				 \includegraphics[height=6 cm]{images/anon_1.png}
			 \end{tabular}
		 \end{center}
 	\end{figure}

\end{frame}

\begin{frame}
	\frametitle{caso di studio: Reid [2011]}
	\framesubtitle{rete utenti imperfetta $\tilde{\mathcal{U}}$ }

	% occorre fare preprocessing per costruire la rete utenti

	% non sappiamo ancora chi siano gli utenti: ma come ammesso dallo stesso Satoshi, è estremamente probabile che
	% in una transazione con più di un input, tutte le chiavi {pk}_input appartengano allo stesso soggetto 
	% vertice diamante == chiave pubblica pk
	% adiacenza tra diamanti == supposizione di medesima proprietà
	% arco diretto == flusso di BTC da una pk a pk'
	
	% abbiamo iniziato ad accomunare delle chiavi pubbliche, ma possiam far di
	% più..

	\begin{figure}[H]
	 	\begin{center}
			 \begin{tabular}{c @{\hspace{1em}} c}
				 \includegraphics[height=6 cm]{images/anon_2.png}
			 \end{tabular}
		 \end{center}
 	\end{figure}
 	

\end{frame}

\begin{frame}
	\frametitle{caso di studio: Reid [2011]}
	\framesubtitle{rete utenti $\mathcal{U}$, rete ancella $\mathcal{A}$}
	
	%	vertice ancella == chiave pubblica 
	%	arco indiretto == paio di chiavi pubbliche in input a una stessa transazione
	% una rete ancella è qualcosa in più di una clique. 
	% 	diametro 4 -> almeno 2 pk dello stesso utente sono connesse indirettamente via 3 transazioni
	
	% IMP nel disegno è graficata un'ancella con ANCHE pks ESTERNE rispetto al sottoesempio
	% in particolare essendo dim(ancella u1)=16 => 
	
	% ogni componento connesso MASSIMALE costituisce un utente -> macrovertice cerchio
	
	% macrovertice cerchio == utente fisico
	% arco diretto == flusso di BTC da un utente a un altro
	
	% U, a differenza di T, non è più un DAG, ma può avere multiarchi e cicli 
	
	\begin{figure}[H]
	 	\begin{center}
			 \begin{tabular}{c @{\hspace{1em}} c}
				 \includegraphics[height=6 cm]{images/anon_3.png}
			 \end{tabular}
		 \end{center}
 	\end{figure}

\end{frame}

\begin{frame}
	\frametitle{caso di studio: Reid [2011]}
	\framesubtitle{integrazione con informazioni esterne}
	
	% SAY forums, tweets
	
	\begin{itemize}
		\item dimensione $\propto |\{K_{PB}\}|$ utente
		\item colore $\propto$ \bitcoinA \;scambiati
	\end{itemize} 
	
	
	\begin{figure}[H]
	 	\begin{center}
			 \begin{tabular}{c @{\hspace{1em}} c}
				 \includegraphics[height=6 cm]{images/anon_4.png}
			 \end{tabular}
		 \end{center}
 	\end{figure}

\end{frame}
\subsection{Double spending}
	\begin{frame}
	\frametitle{caso di studio: Karame [2012]}

	tipologia transazione
	\begin{itemize}
	  \item lenta, \textit{e.g.} acquisto ticket eventi
	  		\vspace{2pt}
	  		\newline sicurezza offerta dal mining
	  \vspace{5pt}
	  \item {\color{blue}veloce}, \textit{e.g.} pagamento in negozio
	  		\vspace{2pt}
	  		\newline $\exists$ possibilità di \textit{double spending}
	  		\begin{itemize}
	  			\item tempi scambio $[s]$ $\ll$ tempi validazione $[min]$
	  			\item Bitcoin segue tecnica \textit{struzzo}
	  			\item problema non grave ma aperto
	  		\end{itemize}
	\end{itemize}

\end{frame}
% ----------------------------------------------------------------------------------------------------------------------------
% \begin{frame}
% 	\frametitle{caso di studio: Karame [2012]}
% 	\framesubtitle{garantire validità nel caso \textit{veloce}}
% 	
% 	\begin{figure}[H]
% 	 	\begin{center}
% 			 \begin{tabular}{c @{\hspace{1em}} c}
% 				 \includegraphics[height=5.5 cm]{images/dspending_1.png}
% 			 \end{tabular}
% 		 \end{center}
%  	\end{figure}
% 
% \end{frame}
% ----------------------------------------------------------------------------------------------------------------------------
\begin{frame}
	\frametitle{caso di studio: Karame [2012]}
	\framesubtitle{ipotesi}

	\begin{columns}
	 \begin{column}{.45\textwidth}
		hosts
		\begin{itemize}
		  \item $A$ peer disonesto
		  \item $H$ complici di $A$ 
		  \item $V$ vendor onesto
		\end{itemize}
		transazioni
		\begin{itemize}
		  \item $\orange{\mathfrak{T}_V}$: acquisto regolare
		  \item $\orange{\mathfrak{T}_A}$: recupero fraudolento
		\end{itemize}
	\end{column}
	
	\begin{column}{.65\textwidth}
		ipotesi
		\begin{itemize}
		  \item $A$ conosce indirizzo IP di $V$
		  \item $\mathcal{C}_A$ trascurabile
		  %SAY 
		  \item $\orange{\mathfrak{I}_V^{in}} = \orange{\mathfrak{I}_A^{in}} \in A$ 
		  \item $ V \ni\, \orange{\mathfrak{I}_V^{out}} \neq \orange{\mathfrak{I}_A^{out}} \in A$
		  \item implementazioni \textit{plain vanilla}
	 	\end{itemize}
 	\end{column}
 	\end{columns}

\end{frame}
% ----------------------------------------------------------------------------------------------------------------------------
\begin{frame}
	\frametitle{caso di studio: Karame [2012]}
	\framesubtitle{idea di massima}
 	
 	\begin{columns}
	 \begin{column}{.61\textwidth}
		\begin{itemize}	 
		  	\item $\orange{\mathfrak{T}_V}, \orange{\mathfrak{T}_A}$ inviate contemporaneamente
		  		\newline  $\;\;\Rightarrow $ incluse nello stesso pool
	  		\item se $\orange{\mathfrak{I}^{in}_\mathfrak{T}}=\orange{\mathfrak{I}^{in}_\mathfrak{T^\prime}}$ 
	  		%$\orange{\mathfrak{T}},\,\orange{\mathfrak{T^\prime}}$ condividono inputs 
	  			\newline $\;\;\Rightarrow $ non ammesse nello stesso pool
	  		\item inclusa solo la prima $\orange{\mathfrak{T}}$ ad arrivare
		  	$\;\;\;\;\;\;\;\;\Rightarrow$
		  	\begin{itemize}
		  		\item $\orange{\mathfrak{T}_A}$ da validare rapidamente
		  		\item $\orange{\mathfrak{T}_V}$ sarà smentita dalla rete
			\end{itemize}
		\end{itemize}
	 \end{column}
	
	 \begin{column}{.6\textwidth}
	 	\begin{figure}[H]
	 	\begin{center}
			 \begin{tabular}{c @{\hspace{1em}} c}
				 \includegraphics[height=5.5 cm]{images/dspending_2.png}
			 \end{tabular}
		 \end{center}
 		\end{figure}
	 \end{column}
	\end{columns}

\end{frame}
% ----------------------------------------------------------------------------------------------------------------------------
\begin{frame}
	\frametitle{caso di studio: Karame [2012]}
	\framesubtitle{$\mathbf{1^a}$ \textbf{condizione}: connessione diretta tra $A$ e $V$}
	
	\fbox{$V$\,riceve prima\,$\orange{\mathfrak{T}_V}$\,di\,$\orange{\mathfrak{T}_A}$}
	\newline oppure\,$V$\,includerebbe prima\,$\orange{\mathfrak{T}_A}$\,nel suo pool
	
	\begin{columns}
		\begin{column}{.5\textwidth}
			\begin{itemize}
			  \item client accetta sempre nuove connessioni < 125 max
			  \item $A$ comunica con $H$
				\begin{itemize}
					\item senza latenza
					\item privatamente    
				\end{itemize}
			  \item $H$ non comunica con $V$
			  \item $A$ invia
				\begin{enumerate}
				  \item $\orange{\mathfrak{T}_V}$ a $V$
				  \item $\orange{\mathfrak{T}_A}$ a $H$
				\end{enumerate}
			\end{itemize}
			 $\;\;\;\;$\fbox{$\Rightarrow \orange{t_V^V} < \orange{t_V^A}$}
		\end{column}
		
		\begin{column}{.60\textwidth}
			\begin{figure}[H]
		 	\begin{center}
				 \begin{tabular}{c @{\hspace{1em}} c}
					 \includegraphics[height=4.75 cm]{images/dspending_3.png}
				 \end{tabular}
			 \end{center}
				\caption{analisi del momento propizio}
	 		\end{figure}
		\end{column}
	\end{columns}		  
\end{frame}
% ----------------------------------------------------------------------------------------------------------------------------
\begin{frame}
	\frametitle{caso di studio: Karame [2012]}
	\framesubtitle{$\mathbf{2^a}$ \textbf{condizione}: diffusione manipolata}
	
	\fbox{$\orange{\mathfrak{T}_A}$ confermata in \textit{blockchain} prima di $\orange{\mathfrak{T}_V}$} 
	\newline oppure $\orange{\mathfrak{T}_A}$ non più validabile
	  		
	\begin{itemize}
	  \item $\text{ogni peer include}\;\orange{\mathfrak{T}_A}\:\dot{\vee}\:\orange{\mathfrak{T}_V}\;\text{in proprio pool}$
	  \begin{itemize}
	  	\item $\orange{\mathfrak{T}_A}, \orange{\mathfrak{T}_V}$ broadcastate in due partizioni 
		\item termine quando $\orange{\mathfrak{T}_A}\:\dot{\vee}\:\orange{\mathfrak{T}_V}$ confermata
 	  \end{itemize}
	  \item \fbox{$\Pr[\orange{\tau_A} < \orange{\tau_V}] \propto \sfrac{\orange{\eta_A}}{\orange{\eta_V}}$} migliora se
		\begin{itemize}
			\item invio di $\orange{\mathfrak{T}_A}$ precede invio di $\orange{\mathfrak{T}_V}$ 
			\item $H$ aiutano $A$ diffondendo $\orange{\mathfrak{T}_A}$ e filtrando $\orange{\mathfrak{T}_V}$ 
		\end{itemize}
% 	  \item ulteriori ipotesi
% 	  	\begin{itemize}
% 	  	  	\item $\exists$ istante ove $\orange{\mathfrak{T}_A},\,\orange{\mathfrak{T}_V}$ convivono
% 	  	  	\item $\forall\,\orange{\varepsilon}\;\mathrm{p.a.p.},\;\;\;
% 	  	  			\Pr[\orange{\tau_A}\sim\mathrm{secs}]\cup\Pr[\orange{\tau_V}\sim\mathrm{secs}]<\orange{\varepsilon}$ 
%   			\item $\orange{\eta_A},\,\orange{\eta_V}$ non scambiano blocchi risolti $\Rightarrow \orange{\tau_A},\,\orange{\tau_V}$ indipendenti
% 	  	\end{itemize}
	\end{itemize}

\end{frame}
% ----------------------------------------------------------------------------------------------------------------------------
\begin{frame}
	\frametitle{caso di studio: Karame [2012]}
	\framesubtitle{probabilità di successo}

	$\Pr[\text{successo in tempo}\;\orange{\delta t}] \sim \mathrm{Bernoulli}(\orange{\eta_A},\,\orange{p})$
	\begin{itemize}
	  \item $\orange{\eta_A}=$ \# peers coinvolti
	  \item $\orange{p}=\Pr[\,\text{peer generi}\:\orange{\mathfrak{B}}\:\text{in}\;{\orange{\delta t}}\,]$ 
	\end{itemize}

	\begin{figure}[H]
	 	\begin{center}
			 \begin{tabular}{c @{\hspace{1em}} c}
				 \includegraphics[height=4.5 cm]{images/dspending_4.png}
			 \end{tabular}
		 \end{center}
		 \caption{$\Pr[\mathrm{successo}\;|\;\delta t=\orange{10s},\;\eta=\orange{6\cdot10^4}]$}
 	\end{figure}
 	
\end{frame}
	%TODO eventualmente \begin{frame}
	\frametitle{Forking}

\end{frame}
	%GRINGO \begin{frame}
	\frametitle{Approfondimenti}
	\begin{itemize}
	  \item falsi miti \\
		\url{http://en.bitcoin.it/wiki/Myths}	  
	  \item frequently asked question \\
		\url{http://bitcoin.org/en/faq}	  
	  \item papers \\
	  	\url{http://en.bitcoin.it/wiki/Research}
	\end{itemize}
\end{frame}


% -------------------------------------------------------------------------
\end{document}